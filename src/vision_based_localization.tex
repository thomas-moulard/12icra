\section{Vision-Based Localization}\label{sec:localization}
Our novel vision-based localization approach combines efficient
monocular vision-based localization techniques with visibility
prediction and stereo visual odometry, exploiting all the vision
capabilities of the HRP-2 robot.

Once we have computed a 3D map of the environment by means of the
described visual SLAM technique, we can use that map for fast and
robust localization. In this context we employ the visibility
prediction technique described in~\cite{Alcantarilla11icra} to perform
an efficient data association between known 3D points and detected 2D
features. This technique has been proved successfully for monocular
vision-based localization in office-like environments
in~\cite{Alcantarilla10icra}.

Now, we will describe how to perform an efficient visibility
prediction of known 3D points and our overall vision-based
localization framework.

%*******************************************************************************************
%*******************************************************************************************
\subsection{Visibility Prediction of known 3D Points}\label{sec:visibility}
Visibility prediction is a commonly used
technique~\cite{Alcantarilla11icra} to greatly reduce the ambiguities
and speed up the data association by making an accurate and robust
prediction of the most likely visible 3D points for a given camera
pose. More specifically, in the visibility prediction problem, we are
interested in the posterior distribution of the visibility $v_{j}$ for
a certain 3D point $x_{j}$ given the query camera pose $\theta$,
denoted as $P(v_{j} \vert \theta)$.

We take the visibility prediction approach described
in~\cite{Alcantarilla11icra} as the basis for our vision-based
localization algorithm. In the mentioned work, they describe how the
visibility of known 3D points can be approximated by using a form of
lazy and memory-based learning technique known as \textit{Locally
  Weighted Learning}~\cite{Atkeson97ai}. This technique is a simple
memory-based classification algorithm and can be implemented very
efficiently. The idea is very simple: given the training data that
consists of a set of reconstructed camera poses $\Theta =
\left\{\theta_1 \ldots \theta_N \right\}$, the 3D point cloud $X =
\left\{x_1 \ldots x_M\right\}$ and a query camera pose $\theta$, we
form a locally weighted average at the query point and take that as an
estimate for $P(v_{j} \vert \theta)$ as follows:
%
\begin{equation} \label{eq:locally_weighted}
 P(v_j \vert \theta) \approx \frac{\sum\limits_{i=1}^{N} k(\theta,\theta_{i})\cdot v_j(\theta_i)}{\sum\limits_{i=1}^{N} k(\theta,\theta_{i})}
\end{equation}
% where the function $k(\theta,\theta_{i})$ is a kernel function that
measures the similarity between two camera poses, and the function
$v_{j}(\theta_i)$ just assigns a real value equal to 1 for those cases
where a certain 3D point $x_{j}$ is visible by a camera pose
$\theta_{i}$ and 0 otherwise. In the end, the main problem is finding
an appropriate kernel function $k(\theta,\theta_{i})$ that captures
correctly the similarity between two camera poses, emphasizing similar
ones and deemphasizing very different camera poses.

The kernel function is learnt by combining the Gaussian kernel and
Mahalanobis distance as described in~\cite{Alcantarilla11icra}. More
in detail, we need to learn the kernel parameters from the training
data, by fitting the kernel function to a set of target values. These
target values $y_{ij}$ are defined as the mean of the ratios between
the intersection of the common 3D points with respect to the number of
3D points visible to each of the two cameras:
%
\begin{equation}\label{eq:similarity_weighted}
y_{ij} = \frac{1}{2}\cdot \left| \frac{\left|X_i \cap X_j \right|}{\left|X_i\right|} + \frac{\left|X_j \cap X_i\right|}{\left|X_j\right|} \right|
\end{equation}
%
Finally, the expression of the kernel function that measures the similarity between two camera poses is:
%
\begin{equation}\label{eq:visibility_metric}
 k_{ij} \equiv k(\vec{\theta}_i,\vec{\theta}_j) =\exp\left(-\left\| \mathbf{A}(\vec{\theta}_i-\vec{\theta}_j)\right\|_{2}\right)
\end{equation}
%
where $\mathbf{A}$ is a $n \times n$ matrix, being $n$ the number of
cues used in the proposed metric. In this work, each camera pose is
parametrized by means of a vector $\vec{\theta}_i =
\left\{T_{i},R_{i}\right\}$ (3D vector for the translation and 4D unit
quaternion for the rotation). For simplicity, we just use two cues in
the proposed metric: difference in camera translation and dot product
between cameras viewing directions vectors, capturing efficiently the
differences between camera poses due to changes in translation and
orientation.

As explained in~\cite{Alcantarilla11icra}, the visibility posterior
can be approximated by just considering the K Nearest Neighbors (KNNs)
of the current query pose $\theta_{t}$. As a consequence, once we find
the KNNs of the current query pose, we only need to predict the
visibilities for the subset of map elements which are at least seen
once by these KNNs. Then, we can set the visibilities to be zero for
the rest of map elements. Finally, we obtain the locally weighted $K$
nearest neighbor approximation for the visibility posterior as
follows:
%
\begin{equation} \label{eq:KNNVisibility}
P(v_{j}=1|\theta) \approx \frac{\sum\limits_{i=1}^{K}k(\theta,\theta_{i}^{v_{j}=1})}{\sum\limits_{i=1}^{K}k(\theta,\theta_{i})}
\end{equation}
%
where only the nearest $K$ samples of the query pose
$\Theta^{K}=\left\{\theta_{1} \ldots \theta_{k}\right\}$ are
considered.

%*******************************************************************************************
%*******************************************************************************************
\subsection{Localization Algorithm}\label{sec:localization_algorithm}
Our localization framework is composed of two different modules:
initialization (re-localization) and a combination between
vision-based localization with visibility prediction and stereo visual
odometry. Now, we will describe each of the different modules.

%*******************************************************************************************
%*******************************************************************************************
\subsubsection{Initialization (Re-Localization)}
During the initialization, the robot can be located in any particular
area of the map. Therefore, we need to find a prior camera pose to
initialize the vision-based localization algorithm. For this purpose,
we compute the appearance descriptors of the detected 2D features in
the new image and match this set of descriptors against the set of
descriptors from the list of stored keyframes from the previous 3D
reconstruction. In the matching process between the two frames, we
perform a RANSAC procedure forcing epipolar geometry constraints. We
recover the camera pose from the stored keyframe that obtains a higher
inliers ratio score. If this inliers ratio is lower than a certain
threshold, we do not initialize the localization algorithm until the
robot moves into a known area yielding a high inliers ratio. At this
point, we are confident about the camera pose prior and initialize the
localization process with the camera pose parameters of the stored
keyframe with the highest score.

Eventually, it may happen that the robot gets lost due to bad
localization estimates or robot kidnapping situations. In those cases,
we perform a fast re-localization by checking the set of appearance
descriptors of the robot's new image against only the stored set of
descriptors of the keyframes that are located in a certain distance
area of confidence centered in the last accepted camera pose estimate.

%*******************************************************************************************
%*******************************************************************************************
\subsubsection{Vision-Based Localization}
Given a prior map of 3D points and perceived 2D features in the image,
our problem to solve is the estimation of the camera pose with respect
to the world coordinate frame. Once the system has a good
initialization, the vision-based localization system works through the
following steps:
%
\begin{enumerate}
\item[i] While the robot is moving, the stereo pair acquires a new set of images from which the disparity map is computed.
\item[ii] A set of image features $Z_{t}=\{z_{t,1} \ldots z_{t,n}\}$ are detected by Harris corner detector only in the left image. Then, a feature descriptor is computed for each of the detected features.
\item[iii] Then, by using the visibility predicition algorithm, a promising subset of highly visible 3D map points is chosen and re-projected onto the image plane based on the estimated previous camera pose $\theta_{t-1}$ and known camera parameters.
\item[iv] Afterwards, a set of putative matches $C_{t}$ are formed where the i-th putative match $C_{t,i}$ is a pair $\{z_{t,k},x_{j}\}$ which comprises of a detected feature $z_{k}$ and a map element $x_{j}$. A putative match is created when the Euclidean distance between the appearance descriptors of a detected feature and a re-projected map element is lower than a certain threshold.
\item[v] Finally, we solve the pose estimation problem minimizing the following cost error function, given the set of putative matches $C_{t}$:
%
\begin{equation} \label{eq:pose_estimation}
\argmin \limits_{\emph{R},\mathbf{t}} \sum \limits_{i=1}^{m} \left\|z_{i} - K \left(\emph{R}\cdot x_{i} + \mathbf{t} \right)\right\|_{2}
\end{equation}
%
\end{enumerate}

where $z_{i}=\left(u_{L},v_{L}\right)$ is the 2D image location of a
feature in the left camera, $x_{i}$ represents the coordinates of a 3D
point in the global coordinate frame, $K$ is the left camera
calibration matrix, and $R$ and $t$ are respectively the rotation and
the translation of the left camera with respect to the global
coordinate frame. The pose estimation problem is formulated as a
nonlinear least squares procedure using the Levenberg-Marquardt
algorithm. The set of putative matches may contain outliers, therefore
RANSAC is used in order to obtain a robust model free of outliers.

There can be some frames where the pose estimation problem cannot be
solved efficiently since we may have textureless areas or slightly
different viewpoints from the ones captured at the mapping
sequence. For those situations, we employ stereo visual odometry to
update the pose of the robot with respect to the map coordinate
frame. We match the set of features between two consecutive steps and
compute the incremental pose as described in
Section~\ref{sec:visual_odometry}. Notice here that our system does
not suffer from the typical drift of visual odometry systems, since in
the next frame the system will try to localize with respect to the
prior 3D map using visibility prediction.

When the number of consecutive frames where the pose estimation
problem fails is higher than a fixed threshold (e.g. 100 frames), we
declare that the tracking is lost and start a re-localization process
based on appearance information.
