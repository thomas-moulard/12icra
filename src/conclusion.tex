\section{Future Work and Conclusions}\label{sec:conclusions}

In this article, we demonstrated that vision-based localization allows
humanoid robots to achieve complex tasks. By integrating the vision
algorithm into the control framework, we have been able to gradually
reshape the trajectory in order to compensate for execution
errors. This allowed the robot to achieve complex tasks which would be
difficult to realize otherwise.

However, several intesresting additions remain to be
integrated. First, the local trajectory modifications are not checked
to make sure that no auto-collision occurs. Currently, a conservative
maximum correction (i.e.\ 2 cm every two steps) is imposed for
safety. By applying recent works such as fast feasibility tests
described by~\citet{Perrin10icra}, we would be able to increase the
maximum correction without compromising the robot safety. Second, the
HRP-2 robot embeds an Inertial Measurement Unit which could be used to
estimate the robot chest attitude. It would definitely be interesting
to provide an initial estimation of the robot motion to help the
localization process. In addition, we are interested in improving the
capabilities of our visual SLAM and vision-based localization systems
towards the goal of long-term localization to deal with possible
changes in the environment.

%%% Local Variables:
%%% ispell-local-dictionary: "american"
%%% LocalWords:  odometry HRP
%%% End:
