\documentclass[conference]{IEEEtran}
\usepackage{times}

% numbers option provides compact numerical references in the text. 
\usepackage[numbers]{natbib}
\usepackage{multicol}
\usepackage[bookmarks=true]{hyperref}
\usepackage{color}
\usepackage{amsmath}

% Commands and other stuff
\DeclareMathOperator*{\argmin}{arg\,min}
\newcommand{\PFA}[1]{{\color{red}\fbox{PFA:} #1}}

\begin{document}

% paper title
\title{Template paper for the \\Robotics: Science and Systems Conference}

% You will get a Paper-ID when submitting a pdf file to the conference system
\author{Author Names Omitted for Anonymous Review. Paper-ID [add your ID here]}

%\author{\authorblockN{Michael Shell}
%\authorblockA{School of Electrical and\\Computer Engineering\\
%Georgia Institute of Technology\\
%Atlanta, Georgia 30332--0250\\
%Email: mshell@ece.gatech.edu}
%\and
%\authorblockN{Homer Simpson}
%\authorblockA{Twentieth Century Fox\\
%Springfield, USA\\
%Email: homer@thesimpsons.com}
%\and
%\authorblockN{James Kirk\\ and Montgomery Scott}
%\authorblockA{Starfleet Academy\\
%San Francisco, California 96678-2391\\
%Telephone: (800) 555--1212\\
%Fax: (888) 555--1212}}


% avoiding spaces at the end of the author lines is not a problem with
% conference papers because we don't use \thanks or \IEEEmembership


% for over three affiliations, or if they all won't fit within the width
% of the page, use this alternative format:
% 
%\author{\authorblockN{Michael Shell\authorrefmark{1},
%Homer Simpson\authorrefmark{2},
%James Kirk\authorrefmark{3}, 
%Montgomery Scott\authorrefmark{3} and
%Eldon Tyrell\authorrefmark{4}}
%\authorblockA{\authorrefmark{1}School of Electrical and Computer Engineering\\
%Georgia Institute of Technology,
%Atlanta, Georgia 30332--0250\\ Email: mshell@ece.gatech.edu}
%\authorblockA{\authorrefmark{2}Twentieth Century Fox, Springfield, USA\\
%Email: homer@thesimpsons.com}
%\authorblockA{\authorrefmark{3}Starfleet Academy, San Francisco, California 96678-2391\\
%Telephone: (800) 555--1212, Fax: (888) 555--1212}
%\authorblockA{\authorrefmark{4}Tyrell Inc., 123 Replicant Street, Los Angeles, California 90210--4321}}

\maketitle

%*******************************************************************************************
%*******************************************************************************************

\begin{abstract}
The abstract goes here.
\end{abstract}

%*******************************************************************************************
%*******************************************************************************************

\IEEEpeerreviewmaketitle

%*******************************************************************************************
%*******************************************************************************************
% INTRODUCTION
\section{Introduction}~\label{sec:introduction}

%*******************************************************************************************
%*******************************************************************************************
% RELATED WORK
\section{Related Work}\label{sec:related}

%*******************************************************************************************
%*******************************************************************************************
% STEREO SLAM
\section{Stereo Simultaneous Localization and Mapping}\label{sec:vslam}
In this section, we briefly review the main components of our stereo visual SLAM system. Notice here that we are interested in using visual SLAM for building a persistent 3D map of the environment that can be used later for vision-based localization and planning~\PFA{is planning the best word here?} purposes. We employ the two cameras that are attached to the ears of the HRP-2 robot. These two cameras have a baseline of approximately 14.4~cm and an horizontal field of view of $90^{\circ}$ for each of the cameras. 

Prior to any SLAM processing, the stereo rig is calibrated obtaining the intrinsics parameters of each of the cameras and the extrinsics parameters between them. Once we have obtained the intrinsics and extrinsics of the stereo rig, we can correct the distortion of the images and perform stereo rectification~\citep{Hartley99ijcv}. Stereo rectification simplifies considerably the stereo correspondences problem and allows to compute dense disparity or depth maps. 

Our visual SLAM system combines accurate relative camera pose estimation by means of visual odometry~\cite{Kaess09icra} and a hierarchical optimization of the motion and the structure by means of local bundle adjustment~\cite{Mouragnon09ivc}. 

%*******************************************************************************************
%*******************************************************************************************
\subsection{Stereo Visual Odometry}\label{sec:visual_odometry}
We estimate the relative camera motion between consecutive frames by matching the set of correspondences between two frames. The set of 2D features are detected by means of the well-known Harris corner detector~\cite{Harris88avc} at the original image resolution. We detect features only for the left image of the stereo pair. Then, we find the correspondences of the 2D features in the right image by accessing the disparity map. At the end, what we have is a set of stereo features $\mathcal{F}_{t}=\left\{\left(u_{L},u_{R},v\right)_{i}\right\}$, where
$\left(u_{L},v\right)$ is the location of the feature in the left image and $\left(u_{R},v\right)$ is the corresponding location in the right image. In addition, we also store for each stereo feature $\mathcal{F}_{t}$ the coordinates of the i-th reconstructed 3D point $h_{i,t}=\left(x \ y \ z\right)^{t}$ with respect to the camera coordinate frame at that time instant $t$. 

For each detected 2D feature in the left image we also extract a descriptor vector that encodes its appearance information. Similar to Speeded Up Robust Features (SURF)~\cite{Bay08cviu}, for a detected feature at a certain scale, we compute a unitary descriptor vector of dimension $16$ in order to speed up the descriptor computation. We use the upright version of the descriptors (no invariance to rotation) since upright descriptors perform better in scenarios where the camera only rotates around its vertical axis, which is often the case of humanoid robots. For simplicity, we do not use any kind of spatial or Gaussian weighting. 

Once we have computed the features descriptors, we find the set of putative matches between the stereo features from the current frame~$\mathcal{F}_{t}$ and the previous one~$\mathcal{F}_{t-1}$ by matching their associated list of descriptors vectors. After finding the set of putative matches between two consecutive frames we estimate the relative camera motion using a standard two-point algorithm in a Random Sample Consensus (RANSAC)~\cite{Bolles81ijcai} setting by minimizing the following cost function:
%
\begin{equation} \label{eq:three_pt}
\argmin_{\textit{R}_{t-1}^{t},\mathbf{t}_{t-1}^{t}} \sum\limits_{i} \left\| z_{i,t} - \Pi\left(\textit{R}_{t-1}^{t},\mathbf{t}_{t-1}^{t},h_{i,t-1}\right)\right\|_{2}
\end{equation}
%
where $z_{i,t}=\left\{\left(u_{L},u_{R},v\right)_{i}\right\}$ are the set of 2D measurements of a stereo feature at time t and $\Pi$ is a function that projects a 3D point $h_{i,t-1}$ (referenced to the camera coordinate frame at time $t-1$) to the image coordinate frame at time $t$. This projection function $\Pi$ involves a rotation $\textit{R}_{t-1}^{t}$ and a translation $\mathbf{t}_{t-1}^{t}$ of 3D points between both coordinate frames and a projection onto the image plane by means of the stereo rig calibration
parameters. The resulting relative camera motion is transformed to a global coordinate frame (usually referenced to the first frame of the sequence) and then is used by the mapping management module. We use the
Levenberg-Marquardt algorithm for all the nonlinear optimizations.

%*******************************************************************************************
%*******************************************************************************************
% MAPPING - BUNDLE ADJUSTMENT
\subsection{Bundle Adjustment}\label{sec:ba}
When the accumulated motion in translation or rotation from the visual odometry module is higher than a fixed threshold we decide to create a new \textit{keyframe}. This keyframe, will be optimized later in a local bundle adjusment procedure. In the local bundle adjustment optimization, 3D points and camera poses are refined simultaneously through the sequence. Similar to~\cite{Mouragnon09ivc} we use a sliding window approach taking into account the last $N$ keyframes, optimizing only $n$ keyframes at each stage. Typical values for $\left(N,n\right)$ are $\left(10,3\right)$ respectively.

We perform an intelligent management of features into the map, in order to produce an equal distribution of feature locations over the image. While adding a new feature to the map, we also store its associated appearance descriptor and 3D point location. Then, we try to match the feature descriptor against the detected new 2D features on a new keyframe by matching their associated descriptors in a high probability search area. In this way, we can create for a map element, \textit{feature tracks} that contain the information of the 2D measurements of the feature (both in left and right views) in several keyframes. Then, this information is used as an input for the local bundle adjustment procedure. Features are deleted from the map when the mean re-projection error per frame in the 3D reconstruction is higher than a fixed threshold (e.g. 3 pixels). 

By means of appearance based methods, loop closure situations can be detected and the residual error in the 3D reconstruction can be corrected in a global BA procedure.

%*******************************************************************************************
%*******************************************************************************************
% LOCALIZATION FRAMEWORK
\section{Vision-Based Localization}\label{sec:localization}
Our novel vision-based localization approach combines efficient monocular vision-based localization techniques with visibility prediction and stereo visual odometry, exploiting all the vision capabilities of the HRP-2 robot. 

Once we have computed a 3D map of the environment by means of the described visual SLAM technique, we can use that map for fast and robust localization. In this context we employ the visibility prediction technique described in~\cite{Alcantarilla11icra} to perform an efficient data association between known 3D points and detected 2D features. This technique has been proved successfully for monocular vision-based localization in office-like environments in~\cite{Alcantarilla10icra}. 

Now, we will describe how to perform an efficient visibility prediction of known 3D points and our overall vision-based localization framework.

%*******************************************************************************************
%*******************************************************************************************
\subsection{Visibility Prediction of known 3D Points}\label{sec:visibility}
Visibility prediction is a commonly used technique~\cite{Alcantarilla11icra} to greatly reduce the ambiguities and speed up the data association by making an accurate and robust prediction of the most likely visible 3D points for a given camera pose. More specifically, in the visibility prediction problem, we are interested in the posterior distribution of the visibility $v_{j}$ for a certain 3D point $x_{j}$ given the query camera pose $\theta$, denoted as $P(v_{j} \vert \theta)$.

We take the visibility prediction approach described in~\cite{Alcantarilla11icra} as the basis for our vision-based localization algorithm. In the mentioned work, they describe how the visibility of known 3D points can be approximated by using a form of lazy and memory-based learning technique known as \textit{Locally Weighted Learning}~\cite{Atkeson97ai}. This technique is a simple memory-based classification algorithm and can be implemented very efficiently. The idea is very simple: given the training data that consists of a set of reconstructed camera poses $\Theta = \left\{\theta_1 \ldots \theta_N \right\}$, the 3D point cloud $X = \left\{x_1 \ldots x_M\right\}$ and a query camera pose $\theta$, we form a locally weighted average at the query point and take that as an estimate for $P(v_{j} \vert \theta)$ as follows:
%
\begin{equation} \label{eq:locally_weighted}
 P(v_j \vert \theta) \approx \frac{\sum\limits_{i=1}^{N} k(\theta,\theta_{i})\cdot v_j(\theta_i)}{\sum\limits_{i=1}^{N} k(\theta,\theta_{i})}
\end{equation}
%
where the function $k(\theta,\theta_{i})$ is a kernel function that measures the similarity between two camera poses, and the function $v_{j}(\theta_i)$ just assigns a real value equal to 1 for those cases where a certain 3D point $x_{j}$ is visible by a camera pose $\theta_{i}$ and 0 otherwise. In the end, the main problem is finding an appropriate kernel function $k(\theta,\theta_{i})$ that captures correctly the similarity between two camera poses, emphasizing similar ones and deemphasizing very different camera poses. 

The kernel function is learnt by combining the Gaussian kernel and Mahalanobis distance as described in~\cite{Alcantarilla11icra}. More in detail, we need to learn the kernel parameters from the training data, by fitting the kernel function to a set of target values. These target values $y_{ij}$ are defined as the mean of the ratios between the intersection of the common 3D points with respect to the number of 3D points visible to each of the two cameras:
%
\begin{equation}\label{eq:similarity_weighted}
y_{ij} = \frac{1}{2}\cdot \left| \frac{\left|X_i \cap X_j \right|}{\left|X_i\right|} + \frac{\left|X_j \cap X_i\right|}{\left|X_j\right|} \right|
\end{equation}
%
Finally, the expression of the kernel function that measures the similarity between two camera poses is:
%
\begin{equation}\label{eq:visibility_metric}
 k_{ij} \equiv k(\vec{\theta}_i,\vec{\theta}_j) =\exp\left(-\left\| \mathbf{A}(\vec{\theta}_i-\vec{\theta}_j)\right\|_{2}\right)
\end{equation}
%
where $\mathbf{A}$ is a $n \times n$ matrix, being $n$ the number of cues used in the proposed metric. In this work, each camera pose is parametrized by means of a vector $\vec{\theta}_i = \left\{T_{i},R_{i}\right\}$ (3D vector for the translation and 4D unit quaternion for the rotation). For simplicity, we just use two cues in the proposed metric: difference in camera translation and dot product between cameras viewing directions vectors, capturing efficiently the differences between camera poses due to changes in translation and orientation.

As explained in~\cite{Alcantarilla11icra}, the visibility posterior can be approximated by just considering the K Nearest Neighbors (KNNs) of the current query pose $\theta_{t}$. As a consequence, once we find the KNNs of the current query pose, we only need to predict the visibilities for the subset of map elements which are at least seen once by these KNNs. Then, we can set the visibilities to be zero for the rest of map elements. Finally, we obtain the locally weighted $K$ nearest neighbor approximation for the visibility posterior as follows:
%
\begin{equation} \label{eq:KNNVisibility}
P(v_{j}=1|\theta) \approx \frac{\sum\limits_{i=1}^{K}k(\theta,\theta_{i}^{v_{j}=1})}{\sum\limits_{i=1}^{K}k(\theta,\theta_{i})}
\end{equation}
%
where only the nearest $K$ samples of the query pose $\Theta^{K}=\left\{\theta_{1} \ldots \theta_{k}\right\}$ are considered. 

%*******************************************************************************************
%*******************************************************************************************
\subsection{Localization Algorithm}\label{sec:localization_algorithm}
Our localization framework is composed of two different modules: initialization (re-localization) and a combination between vision-based localization with visibility prediction and stereo visual odometry. Now, we will describe each of the different modules.

%*******************************************************************************************
%*******************************************************************************************
\subsubsection{Initialization (Re-Localization)}
During the initialization, the robot can be located in any particular area of the map. Therefore, we need to find a prior camera pose to initialize the vision-based localization algorithm. For this purpose, we compute the appearance descriptors of the detected 2D features in the new image and match this set of descriptors against the set of descriptors from the list of stored keyframes from the previous 3D reconstruction. In the matching process between the two frames, we perform a RANSAC procedure forcing epipolar geometry constraints. We recover the camera pose from the stored keyframe that obtains a higher inliers ratio score. If this inliers ratio is lower than a certain threshold, we do not initialize the localization algorithm until the robot moves into a known area yielding a high inliers ratio. At this point, we are confident about the camera pose prior and initialize the localization process with the camera pose parameters of the stored keyframe with the highest score.

Eventually, it may happen that the robot gets lost due to bad localization estimates or robot kidnapping situations. In those cases, we perform a fast re-localization by checking the set of appearance descriptors of the robot's new image against only the stored set of descriptors of the keyframes that are located in a certain distance area of confidence centered in the last accepted camera pose estimate.

%*******************************************************************************************
%*******************************************************************************************
\subsubsection{Vision-Based Localization}
Given a prior map of 3D points and perceived 2D features in the image, our problem to solve is the estimation of the camera pose with respect to the world coordinate frame. Once the system has a good initialization, the vision-based localization system works through the following steps:
%
\begin{enumerate}
\item[i] While the robot is moving, the stereo pair acquires a new set of images from which the disparity map is computed.
\item[ii] A set of image features $Z_{t}=\{z_{t,1} \ldots z_{t,n}\}$ are detected by Harris corner detector only in the left image. Then, a feature descriptor is computed for each of the detected features. 
\item[iii] Then, by using the visibility predicition algorithm, a promising subset of highly visible 3D map points is chosen and re-projected onto the image plane based on the estimated previous camera pose $\theta_{t-1}$ and known camera parameters.
\item[iv] Afterwards, a set of putative matches $C_{t}$ are formed where the i-th putative match $C_{t,i}$ is a pair $\{z_{t,k},x_{j}\}$ which comprises of a detected feature $z_{k}$ and a map element $x_{j}$. A putative match is created when the Euclidean distance between the appearance descriptors of a detected feature and a re-projected map element is lower than a certain threshold. 
\item[v] Finally, we solve the pose estimation problem minimizing the following cost error function, given the set of putative matches $C_{t}$:
%
\begin{equation} \label{eq:pose_estimation}
\argmin \limits_{\emph{R},\mathbf{t}} \sum \limits_{i=1}^{m} \left\|z_{i} - K \left(\emph{R}\cdot x_{i} + \mathbf{t} \right)\right\|_{2}  
\end{equation}
%
\end{enumerate}
where $z_{i}=\left(u_{L},v_{L}\right)$ is the 2D image location of a feature in the left camera, $x_{i}$ represents the coordinates of a 3D point in the global coordinate frame, $K$ is the left camera calibration matrix, and $R$ and $t$ are respectively the rotation and the translation of the left camera with respect to the global coordinate frame. The pose estimation problem is formulated as a nonlinear least squares procedure using the Levenberg-Marquardt algorithm. The set of putative matches may contain outliers, therefore RANSAC is used in order to obtain a robust model free of outliers. 

There can be some frames where the pose estimation problem cannot be solved efficiently since we may have textureless areas or slightly different viewpoints from the ones captured at the mapping sequence. For those situations, we employ stereo visual odometry to update the pose of the robot with respect to the map coordinate frame. We match the set of features between two consecutive steps and compute the incremental pose as described in Section~\ref{sec:visual_odometry}. Notice here that our system does not suffer from the typical drift of visual odometry systems, since in the next frame the system will try to localize with respect to the prior 3D map using visibility prediction. 

When the number of consecutive frames where the pose estimation problem fails is higher than a fixed threshold (e.g. 100 frames), we declare that the tracking is lost and start a re-localization process based on appearance information.

%*******************************************************************************************
%*******************************************************************************************
% RESULTS
\section{Experimental Results}\label{sec:results}
The image resolution that we used in our set of experiments was $320 \times 240$.


\subsection{Comparison to MOCAP Data}\label{sec:mocap}
TODO: Show localization results compared to mocap data. Circular sequence.

%*******************************************************************************************
%*******************************************************************************************
% CONCLUSIONS
\section{Future Work and Conclusions}\label{sec:conclusions}

The conclusion goes here.

%*******************************************************************************************
%*******************************************************************************************
\section*{Acknowledgments}

%*******************************************************************************************
%*******************************************************************************************
%% Use plainnat to work nicely with natbib. 
\bibliographystyle{plainnat}
\bibliography{references}

\end{document}